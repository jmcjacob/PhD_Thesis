\section{Introduction to Problem}
\label{sec:intoduction_to_problem}
This is where details about the medical issues and how they can be solved used artificial intelligence. This is where the limitations of AI will be discussed.



\section{Research Contributions}
\label{sec:research_contributions}

In this thesis we propose a pathway used to train medical image analysis systems that can take advantage of unannotated data and produce a system that is both selective and cost sensitive in order to minimise risk used as part of a clinical pipeline. The contributions that form the building blocks for this pathway are as followed:

\begin{itemize}
	\item An active learning framework for histopathology patches that samples larger patches for annotation to increase annotation throughput without adding additional work to annotators.
	
	\item Multi-directional contrastive predictive coding, A unsupervised representation learning algorithm built for medical images with no clear directionality.

	\item A empirical comparison of calibration methods for deep medical image classifiers on histopathology patches and skin lesions images.
	
	\item A comparison of selective classification methods in both binary and multi class settings covering Bayesian neural networks, calibrated neural networks and purpose built models for selective classification.
	
	\item A proposed method for selective classification that makes decisions based on expected costs in situations with asymmetric misclassification costs.
	
	\item Something about Selective Triage
	
	\item Something about dataset generalisation
\end{itemize}



\section{Thesis Structure}
\label{sec:thesis_structure}
These need to be revised.

\subsection*{Annotator Efficient Active Learning}
Methods to reduce the need for costly data annotations become increasingly important as deep learning gains popularity in medical image analysis and digital pathology~\citep{tizhoosh2018artificial}. Active learning is an appealing approach that can reduce the amount of annotated data needed to train machine learning models~\citep{settles2012active}, but traditional active learning strategies do not always work well with deep learning~\citep{wang2016cost}. In patch-based machine learning systems, active learning methods typically request annotations for small individual patches which can be tedious and costly for the annotator who needs to rely on visual context for the patches. We propose an active learning framework that selects regions for annotation that are built up of several patches, which should increase annotation throughput~\citep{carse2019active}. The framework was evaluated with several query strategies on the task of nuclei classification. Convolutional neural networks were trained on small patches, each containing a single nucleus. Traditional query strategies performed worse than random sampling.

\subsection*{Unsupervised Representation Learning}
Digital pathology tasks have benefited greatly from modern deep learning algorithms. However, their need for large quantities of annotated data has been identified as a key challenge. This need for data can be countered by using unsupervised learning in situations where data are abundant but access to annotations is limited. Feature representations learned from unannotated data using contrastive predictive coding (CPC) have been shown to enable classifiers to obtain state of the art performance from relatively small amounts of annotated computer vision data. We present a modification to the CPC framework for use with digital pathology patches. This is achieved by introducing an alternative mask for building the latent context and using a multi-directional PixelCNN autoregressor. To demonstrate our proposed method, we learn feature representations from the Patch Camelyon histology dataset. We show that our proposed modification can yield improved deep classification of histology patches.

\subsection*{Selective Classification}

\subsection*{Asymmetric Risks}

\subsection*{Conclusion}
