\section{Introduction to Problem}
\label{sec:intoduction_to_problem}
Modern deep learning algorithms have been shown to improve performance for medical image analysis tasks such as classification, segmentation, and detection for different medical modalities including digital pathology. However, deep learning algorithms require large, annotated datasets to train models as these models learn a deep feature representation as well as a discriminative classifier. When dealing with medical images, annotation can be expensive as annotating requires specialist training and can be very time intensive. This requirement for annotated data has been identified as a key challenge for using deep learning algorithms for medical image analysis in general~\citep{litjens2017survey}. 


\section{Research Contributions}
\label{sec:research_contributions}

In this thesis we propose a pathway used to train medical image analysis systems that can take advantage of unannotated data and produce a system that is both selective and cost sensitive in order to minimise risk used as part of a clinical pipeline. The contributions that form the building blocks for this pathway are as followed:

\begin{itemize}
	\item An active learning framework for histopathology patches that samples larger patches for annotation to increase annotation throughput without adding additional work to annotators.
	
	\item Multi-directional contrastive predictive coding, A unsupervised representation learning algorithm built for medical images with no clear directionality.

	\item A empirical comparison of calibration methods for deep medical image classifiers on histopathology patches and skin lesions images.
	
	\item A comparison of selective classification methods in both binary and multi class settings covering Bayesian neural networks, calibrated neural networks and purpose built models for selective classification.
	
	\item A proposed method for selective classification that makes decisions based on expected costs in situations with asymmetric misclassification costs.
	
	\item Something about Selective Triage
	
	\item Something about dataset generalisation
\end{itemize}



\section{Thesis Structure}
\label{sec:thesis_structure}
These need to be revised.

\subsection*{Annotator Efficient Active Learning}
With the increasing popularity of deep learning in medical image analysis and digital pathology~\citep{tizhoosh2018artificial}, it has become increasingly crucial to develop methods that can reduce the need for costly data annotations. Active learning is a promising approach to minimize the amount of annotated data required to train machine learning models~\citep{settles2012active}. However, the effectiveness of traditional active learning strategies with deep learning is limited~\citep{wang2016cost}. In patch-based machine learning systems, active learning methods typically request annotations for individual small patches, which can be laborious and expensive for annotators who must rely on visual context. To address this issue, we propose an active learning framework that selects regions for annotation that are composed of multiple patches, which is expected to increase annotation throughput~\citep{carse2019active}. We evaluated the framework with various query strategies on the task of nuclei classification, using convolutional neural networks trained on small patches containing single nuclei. Traditional query strategies performed worse than random sampling.

\subsection*{Unsupervised Representation Learning}
Recent advancements in deep learning algorithms have had a significant impact on digital pathology tasks. However, a significant challenge in this field is the need for large amounts of annotated data. To overcome this issue, unsupervised learning techniques, particularly contrastive predictive coding (CPC)~\citep{oord2018representation}, have been proposed to leverage abundant but unannotated data for training classifiers. In this chapter, a modification to the CPC framework for use in digital pathology patch classification is purposed, which involves the use of an alternative mask to construct the latent context and a multi-directional PixelCNN autoregressor~\citep{oord2016pixel}. Using the Path Camelyon histology patch dataset~\citep{veeling2018rotation}, it is demonstrated that this purposed method can produce effective deep feature representations for improved classification accuracy in digital pathology when compared to the standard implementation of CPC~\citep{carse2021unsupervised}.

\subsection*{Predictive Probability Calibration}
It is well established that deploying deep learning classifiers for medical image analysis tasks requires careful consideration of issues related to predictive calibration~\citep{maron2019systematic}. Mis-calibration, defined as the discrepancy between predictive probability (confidence) and classification correctness~\citep{guo2017calibration}, can significantly impact the ability to make cost-sensitive and selective decisions~\citep{carse2021robust}. To understand the effectiveness of various calibration methods, an empirical study was conducted on two medical image datasets: one for multi-class dermatology classification and one for binary histopathology image classification. The study applied the temperature scaling method, in which the temperature parameter is optimized using various calibration measures instead of the standard negative log-likelihood, to networks trained with one-hot encoding and cross-entropy loss, as well as networks trained with focal loss and label smoothing. The results of these methods were compared to those obtained using two Bayesian neural network approaches. The findings suggest that while alternative calibration metrics may not offer significant advantages for tuning temperature, temperature scaling of networks trained with focal loss and appropriate hyperparameters demonstrated strong performance in terms of both calibration and accuracy across both datasets~\citep{carse2022calibration}.

\subsection*{Asymmetrical Selective Classification}
Automated image analysis of skin lesions has potential to improve diagnostic decision making. A clinically useful system should be selective, rejecting images it is ill-equipped to classify, for example because they are of lesion types not represented well in training data. Furthermore, lesion classifiers should support cost-sensitive decision making. We investigate methods for selective, cost-sensitive classification within a binary setting of benign or malignant using test images of lesion types represented and not represented in training data. Further experiments with multi-class selective cost-sensitive classification with misclassification costs provided by clinical dermatologists based on healthcare economics. We experiment with different methods of uncertainty estimation with neural networks and probability calibration.
% Needs editing when experiment results are in.
We introduce EC-SelectiveNet, a modification to SelectiveNet that discards the selection head at test time, making decisions based on expected costs instead. Experiments show that training for full coverage is beneficial even when operating at lower coverage, and that EC-SelectiveNet outperforms standard cross-entropy training, whether temperature scaling or Monte Carlo dropout averaging are used, in both symmetric and asymmetric cost settings.

\subsection*{Cross Site Generalisation}
This chapter examines the generalizability of deep neural network classifiers for macroscopic skin lesion images in the NHS of the UK. Although deep learning has shown promise in dermatology, its ability to accurately diagnose macroscopic skin disease images that lack dermoscopic information remains a significant challenge~\citep{jones2022artificial}. To address this, four diagnostic image datasets were utilized, including two locally-sourced datasets and two publicly available datasets. Two types of neural network models were trained and evaluated on each dataset, with pre-training on the SD-260~\citep{yang2019self} dataset followed by fine-tuning on the target domain data showing the most promising results. This study emphasizes the importance of assessing the generalizability of deep learning algorithms for macroscopic skin lesion images in real-world settings and highlights the potential benefits of utilizing vast public macroscopic datasets for pre-training and fine-tuning. Future research is necessary to evaluate the generalizability of these algorithms across different populations and acquisition settings.

\subsection*{Conclusion}
