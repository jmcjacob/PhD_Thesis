\section{Summary of Contributions}
This thesis represents a contribution to the field of medical image analysis and machine learning, pursuing two primary aims. Firstly, it aims to address the challenge of limited annotated data, particularly for histopathological whole slide images, by utilising active learning and unsupervised learning techniques. Secondly, it seeks to enhance the accuracy of asymmetrical selective classification for skin lesion images. Additionally, the thesis presents secondary contributions in the areas of predictive probability calibration and dataset generalisation.

\subsection{Scarcity of Annotations}
\textit{How can a deep learning model be effectively trained to achieve optimal performance when faced with a scarcity of annotations, and a large corpus of unannotated data?}

To achieve this aim, the first choice of method investigated was active learning, as it is a type of machine learning that seeks to select the most beneficial unannotated data for the model to annotate~\citep{settles2009active}. However, this led to the identification of limitations of active learning and its application to histopathology patches, where nuclei patches may be difficult to annotate. Chapter~\ref{ch:active_learning} introduced an active-learning framework designed to select tiles composed of multiple patches for annotation with a view to making the annotation task easier and thus increasing annotation throughput~\citep{carse2019active}. The efficacy of this framework was evaluated using various query strategies on nuclei classification tasks by employing CNNs trained on small patches containing single nuclei. 

The results suggest that traditional active-learning approaches are less effective when applied to deep-learning models, while specialised active-learning techniques for deep-learning fail to outperform random sampling baselines. This phenomenon has been previously noted in literature on active deep-learning~\citep{ren2021survey} and underscores the need for more robust active-learning methods in this domain. Although this chapter demonstrates that active learning holds promise as a means to address these challenges, further research is required to achieve significant improvements on tasks such as those presented herein. This motivated an investigation into unsupervised learning techniques as a complementary approach.

As discussed in Chapter~\ref{ch:active_learning}, active learning alone may not provide representative enough features for a deep learning model to learn from the annotated data. This leads to the investigation of unsupervised representative learning techniques, which can enable a model to learn generalisable representative features from the unannotated data, which can then be fine-tuned. Chapter~\ref{ch:unsupervised_representation_learning} proposed a modification to the CPC framework~\citep{oord2018representation} for digital pathology patch classification. The modification involved using an alternative infilling-style mask to construct the latent context and a multi-directional PixelCNN autoregressor~\citep{oord2016pixel}.

The experiments conducted to evaluate the proposed modification to the CPC framework revealed that the original implementation of CPC is not well-suited for patch-based digital pathology tasks. However, the proposed multi-directional modifications to the CPC led to better results and improved classification accuracies on transfer learning tasks, where access to annotated data is limited~\citep{carse2021unsupervised}. Thus, the combination of active learning and unsupervised representation learning holds promise in digital pathology tasks.

\newpage
\subsection{Asymmetrical Selective Classification}
\textit{To what extent can selective classification techniques be applied in order to mitigate the costs associated with asymmetrical misdiagnosis of skin lesion images?}

Chapter~\ref{ch:selective_classification} presents an investigation into the efficacy of selective classification as a potential solution to asymmetrical misdiagnosis in skin lesion images. The research aims to address this issue through the exploration of cost-sensitive classification techniques in both binary triage and multi-class disease classification scenarios using a dermatology dataset. The chapter draws on the expertise of clinical dermatologists to provide asymmetrical misclassification costs based on healthcare economic estimations. Methods for uncertainty estimation with neural networks and probability calibration were evaluated, and a novel modification to SelectiveNet~\citep{geifman2019selectivenet}, known as EC-SelectiveNet~\citep{carse2021robust}, was proposed. 

The results suggest that SelectiveNet exhibited inferior performance compared to other selective classification methods, with the exception of when it was trained with a target coverage of $1.0$. In contrast, EC-SelectiveNet, trained with a target coverage of $1.0$, consistently outperformed all other methods in both binary and multi-class settings in the presence of asymmetrical costs. The utilisation of Bayesian neural networks had minimal effect on the predictions when averaged in any setting. Interestingly, the use of various uncertainty measures resulted in different outcomes, with the predictive entropy measure surpassing all others, particularly in an asymmetric setting, while the variational ratios performed poorly in both symmetric and asymmetric cost settings. The results also revealed that temperature scaling for the calibration of predictions to enhance selective classification led to higher costs at higher coverage levels in the context of asymmetrical cost settings. The study evaluated diverse selective classification settings and underscores the need for further research to advance selective classification methods and comprehend their performance in asymmetrical cost settings. These efforts are crucial for the application of classification in clinical settings, where asymmetrical costs are common, and not all images can be classified, necessitating the use of rejection.

\subsection{Secondary Contributions}
\subsubsection{Predictive Probability Calibration}
Calibration denotes the systematic procedure of conforming the anticipated probabilities of a model with the authentic probabilities of the target variable~\citep{guo2017calibration}. Poor calibration performance in modern deep neural networks can hinder the calculation of a model's uncertainty, thereby affecting uncertainty-dependent techniques like active learning and cost-sensitive decision-making~\citep{carse2021robust}. To address this, Chapter~\ref{ch:classification_claibration} presented an empirical investigation of calibration techniques on two medical image classification tasks: multi-class dermatology classification and binary histopathology image classification. The study implemented temperature scaling, optimising the temperature parameter using various calibration measures rather than the standard negative log-likelihood. This method was applied to networks trained with one-hot encoding and cross-entropy loss, as well as networks trained with focal loss and label smoothing. Two Bayesian neural network approaches were also utilised for comparison. The results demonstrated that while alternative calibration metrics may not provide significant advantages for tuning temperature, temperature scaling of networks trained with focal loss and appropriate hyperparameters exhibited robust performance in terms of both calibration and accuracy across both datasets~\citep{carse2022calibration}.

\subsubsection{Dataset Generalisation}
In chapters~\ref{ch:active_learning}, \ref{ch:unsupervised_representation_learning}, \ref{ch:classification_claibration} and \ref{ch:selective_classification}, open-source datasets were utilised to train and evaluate deep learning models. However, it is important to note that although open-source datasets are useful for experimentation, the models produced may not be suitable for clinical use~\citep{wu2022skin}. To address this concern, Chapter~\ref{ch:dataset_generalisation} conducts an investigation to determine the generalisability of models trained with open-source data to locally collected macroscopic datasets from primary care referrals. Two types of neural networks, a CNN and a transformer, were employed to evaluate the model’s generalisation performance on two open-source datasets and two locally collected datasets from the NHS. The findings emphasise the significance of assessing the generalisability of deep learning algorithms for macroscopic skin lesion images in real-world settings. Moreover, the study highlights the potential benefits of utilising large public macroscopic datasets for pre-training and fine-tuning the algorithms.



\section{Limitations and Future Work}
This section describes the limitations of this thesis as well as outlining potential avenues for further research and areas for improvement.

\subsection{Annotator Efficient Active Learning}
Chapter~\ref{ch:active_learning} presented an investigation into improving annotation throughput on deep active learning methods for histopathology patches, with the goal of expanding the volume of annotations gathered while minimising annotation costs. The proposed approach entails simplifying the annotation task to optimise annotator time allocation. Notably, there has been limited research on enhancing annotator efficiency for medical image analysis~\citep{ren2021survey}, and further investigation is warranted across diverse modalities. One possible avenue for exploration involves the application of established methods that address asymmetrical annotation costs, such as CEREALS~\citep{mackowiak2018cereals}, to tasks like whole-slide segmentation, where the costs of obtaining annotations are typically uniform~\citep{budd2021survey}.

The experiments relied on simulated active learning scenarios in which annotations had already been collected and provided automatically upon query. This approach allowed for rapid development of active learning query strategies but did not enable investigation into how a human in the loop would interact with the active learning scenario. Another limitation of the experiments was the dataset used, which was an open-source dataset that had been pre-filtered by selecting the best examples and removing any challenging examples or anomalies that could arise in real-world scenarios where an active learning query system would need to be employed. Therefore, these experiments may not fully capture the complexities and challenges of active learning in real-world settings. Both of these can be mitigated in future by using real unannotated datasets conducting a case study by having clinicians annotate sets of queried data from different query strategies.

The evidence in Chapter~\ref{ch:active_learning} shows that current active learning query strategies don’t perform much better than random querying. This suggests that there is room for improvement in developing new strategies. One promising research direction would be to explore the integration of batch-aware active learning and semi-supervised learning. This could be achieved by designing an active learning query strategy that places emphasis on improving the semi-supervised learning performance. One such attempt has been made for a scoring query strategy like CEAL~\citep{wang2016cost}, which merges softmax response active learning with the semi-supervised method of pseudo labelling. Despite its rudimentary nature, this approach demonstrated encouraging results and is open to further improvement.


\subsection{Unsupervised Representation Learning}
Chapter~\ref{ch:unsupervised_representation_learning} delved into the topic of unsupervised representation learning, an area that has witnessed significant progress and continues to be a subject of active development. Although much of the research in this field has been centred around image datasets such as ImageNet~\citep{deng2009imagenet}, it is important to note that the developed methods may not be optimally suited for medical image datasets. An alternative approach involves disentanglement methods that aim to learn a model capable of identifying and disentangling the underlying factors in the observable data. A recent survey paper by \cite{liu2022learning} provides an overview of such methods. In the context of histopathological patches, disentanglement can facilitate the production of feature representations that capture variations in slide staining, among other factors. These features, when used in conjunction with transfer learning, can yield improved accuracy and generalisation performance on new data.

\subsection{Predictive Probability Calibration}
The experimental findings presented in Chapter~\ref{ch:classification_claibration} indicate that temperature scaling~\citep{guo2017calibration} optimised with any measure of calibration can effectively enhance calibration for multi-class classification tasks. As a post-hoc calibration method, temperature scaling is readily applicable to pre-trained models and does not interfere with the model training process. These results suggest that future research should prioritise the further development and refinement of post-hoc calibration methods. While temperature scaling involves the learning of a single scaling value for model calibration, recent work investigates alternative post-hoc calibration methods that warrant further exploration~\citep{song2021classifier}. In order to assess the performance of these methods, appropriate evaluation metrics must be developed and investigated.

The evaluation of calibration is a crucial aspect in the development of classification systems. Despite its significance, the determination of accurate calibration measures remains a subject of active research, although efforts have been made to incorporate them into various medical image analysis studies~\citep{maier2022metrics}. In Chapter~\ref{ch:classification_claibration}, the KDE-ECE~\citep{zhang2020mix} approach was employed to quantify the calibration of the trained models. This measure evaluates the overall calibration performance across the confidence and correctness distribution. It is noteworthy that KDE-ECE is just one of several calibration measures and should be used in conjunction with other approaches, such as the maximum calibration error, which indicates the maximum calibration error of a bin rather than the weighted average of errors. Furthermore, there is scope for exploring alternative calibration measures, such as those proposed by \cite{nixon2019measuring}.

\subsection{Asymmetrical Selective Classification}
In Chapter~\ref{ch:selective_classification}, experiments are presented on selective classification of skin lesion images with both symmetrical and asymmetrical misdiagnosis costs. The asymmetrical costs utilised in this study were derived from rough estimates provided by a consultant-level dermatologist. However, these costs are subject to variation based on local healthcare economics and are likely to change over time. Consequently, the development of additional asymmetrical costs for realistic settings is necessary for the evaluation of future algorithms in a practical environment.

The findings presented in Chapter~\ref{ch:selective_classification} suggest that both Bayesian and SelectiveNet~\citep{geifman2019selectivenet} models encountered challenges in symmetrical and asymmetrical environments during selective classification experiments. The most effective model was found to be the standard CNN model utilising expected costs, which exhibited good calibration and performance. Temperature scaling~\citep{guo2017calibration}, the only calibration method evaluated, did not meet expectations. However, optimising decision calibration~\citep{zhao2021calibrating} of the models could potentially enhance selective classification performance in asymmetrical misclassification cost environments. This could involve extending asymmetrical decision calibration to facilitate selective classification of skin lesions by incorporating the option to reject an image due to excessive expected loss.

\subsection{Dataset Generalisation}
In Chapter~\ref{ch:dataset_generalisation}, the generalisability of models trained with skin lesion datasets to other datasets is investigated. Based on the findings of this investigation, it is concluded that future research efforts should prioritise the exploration of techniques that can effectively adapt cross-domain models. This should be done by considering the varying costs associated with misdiagnoses of different types, with a particular focus on making cost-sensitive classification decisions~\citep{guan2021domain,carse2021robust}. To improve the accuracy of classification decisions, incorporating well-calibrated classifiers is recommended. This approach would enable the implementation of selective classification decisions. Furthermore, to enhance the performance of deep learning algorithms, it is essential to acquire and annotate data in a prospective manner during dermatology consultant triaging and clinical work. This would ultimately lead to the creation of larger local datasets that would be beneficial in improving the performance of the algorithms. However, it is important to acknowledge that the findings of this study, similar to other studies, are limited by the restricted number of critical diagnostic categories that are examined. Although including more diagnostic categories comes with its own issues known as the long tail problem where the more diseases are included, they are less represented in the training data~\citep{roy2022does}.