\section{Summary of Contributions}
Summary of Contributions



\section{Limitations}
Limitations



\section{Future Work}
In the following section, the future work of this thesis will be discussed, outlining potential avenues for further research and areas for improvement.

\subsection{Annotator Efficient Active Learning}
Chapter~\ref{ch:active_learning} presents an inquiry into the augmentation of annotation throughput in the context of active learning, with the goal of expanding the volume of annotations gathered while minimising annotation costs. The proposed approach entails simplifying the annotation task to optimise annotator time allocation. Notably, there has been limited research on enhancing annotator efficiency for medical image analysis, and further investigation is warranted across diverse modalities. One possible avenue for exploration involves the application of established methods that address asymmetrical annotation costs, such as CEREALS~\citep{mackowiak2018cereals}, to tasks like whole-slide segmentation, where the costs of obtaining annotations are typically uniform~\citep{budd2021survey}.

Moreover, the empirical evidence presented in Chapter~\ref{ch:active_learning} indicate that many of the prevailing query strategies in active learning still encounter significant challenges in achieving superior performance when compared to random querying. These results imply that there is ample opportunity for the development of novel active learning query strategies that can enhance performance. In personal opinion, one promising research direction would be to explore the integration of batch-aware active learning and semi-supervised learning. This could be achieved by designing an active learning query strategy that places emphasis on improving the semi-supervised learning performance. One such attempt has been made for a scoring query strategy like CEAL~\citep{lee2013pseudo}, which merges softmax response active learning with the semi-supervised method of pseudo labelling. Despite its rudimentary nature, this approach demonstrated encouraging results and is susceptible to further improvement.

\subsection{Unsupervised Representation Learning}
Chapter~\ref{ch:unsupervised_representation_learning} delved into the topic of unsupervised representation learning, an area that has witnessed significant progress and continues to be a subject of active development. Although much of the research in this field has been centred around image datasets, such as ImageNet~\citep{deng2009imagenet}, it is important to note that the developed methods may not be optimally suited for medical image datasets. An alternative approach involves disentanglement methods that aim to learn a model capable of identifying and disentangling the underlying factors in the observable data. A recent survey paper by \cite{liu2022learning} provides an overview of such methods. In the context of histopathological patches, disentanglement can facilitate the production of feature representations that capture variations in slide staining, among other factors. These features, when used in conjunction with transfer learning, can yield improved accuracy and generalisation performance on new data.

\subsection{Predictive Probability Calibration}
The experimental findings presented in Chapter~\ref{ch:classification_claibration} indicate that temperature scaling~\citep{guo2017calibration} optimised with any measure of calibration can effectively enhance calibration for multi-class classification tasks. As a post-hoc calibration method, temperature scaling is readily applicable to pre-trained models and does not interfere with the model training process. These results suggest that future research should prioritise the further development and refinement of post-hoc calibration methods. While temperature scaling involves the learning of a single scaling value for model calibration, recent work investigates alternative post-hoc calibration methods that warrant further exploration~\citep{song2021classifier}. In order to assess the performance of these methods, appropriate evaluation metrics must be developed and investigated.

The evaluation of calibration is a crucial aspect in the development of classification systems. Despite its significance, the determination of accurate calibration measures remains a subject of active research, although efforts have been made to incorporate them into various medical image analysis studies~\citep{maier2022metrics}. In Chapter~\ref{ch:classification_claibration}, the KDE-ECE~\citep{zhang2020mix} approach was employed to quantify the calibration of the trained models. This measure evaluates the overall calibration performance across the confidence and correctness distribution. It is noteworthy that KDE-ECE is just one of several calibration measures and should be used in conjunction with other approaches, such as the maximum calibration error, which indicates the maximum calibration error of a bin rather than the weighted average of errors. Furthermore, there is scope for exploring alternative calibration measures, such as those proposed by \cite{nixon2019measuring}.

\subsection{Asymmetrical Selective Classification}
In Chapter~\ref{ch:selective_classification}, experiments are presented on selective classification of skin lesion images with both symmetrical and asymmetrical misdiagnosis costs. The asymmetrical costs utilised in this study were derived from rough estimates provided by a consultant-level dermatologist. However, these costs are subject to variation based on local healthcare economics and are likely to change over time. Consequently, the development of additional asymmetrical costs for realistic settings is necessary for the evaluation of future algorithms in a practical environment.

The results from Chapter~\ref{ch:selective_classification}, which presents various selective classification experiments, indicate that both Bayesian models and SelectiveNet~\citep{geifman2019selectivenet} models encountered difficulties in symmetrical and asymmetrical environments. The most effective model was the standard CNN model utilizing expected costs, which demonstrated good calibration and performance. These findings suggest that enhancing the calibration of models could improve selective classification performance in both symmetrical and asymmetrical environments. Temperature scaling~\citep{guo2017calibration} was the only calibration method evaluated; however, in selective classification, the only differences in predicted probabilities that are significant are those that may result in different decisions being made. This could involve extending decision calibration to facilitate selective classification of skin lesions (i.e., incorporating the option to reject an image due to excessive expected loss) such as in the work from \cite{zhao2021calibrating}.


\subsection{Dataset Generalisation}
Asymmetrical Risks